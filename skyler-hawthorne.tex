%
% LaTeX source of my resume
% =========================
%
% Heavily commented to to fit even LaTeX beginners (hopefully).
%
% See the `README.md` file for more info.
%
% This file is licensed under the CC-NC-ND Creative Commons license.
%


% Start a document with the here given default font size and paper size.
\documentclass[10pt]{article}

% Set the page margins.
\usepackage[margin=0.75in]{geometry}

% Setup the language.
\usepackage[english]{babel}
\hyphenation{Some-long-word}

% Makes resume-specific commands available.
\usepackage{resume}

\usepackage[utf8]{inputenc}

\begin{document}  % begin the content of the document
\sloppy  % this to relax whitespacing in favour of straight margins


% title on top of the document
\maintitle{Skyler Hawthorne}{02 March 1990}{Last update on \today}

\nobreakvspace{0.3em}  % add some page break averse vertical spacing

% \noindent prevents paragraph's first lines from indenting
% \mbox is used to obfuscate the email address
% \sbull is a spaced bullet
% \href well..
% \\ breaks the line into a new paragraph
\noindent\href{mailto:skyler.hawthorne.at.gmail.dot.com}{skylerhawthorne\mbox{}@\mbox{}gmail.com}\sbull
(510) 371-4123\sbull
\href{http://www.linkedin.com/in/skylerhawthorne}{www.linkedin.com/in/skylerhawthorne}
\\
815 Sea Spray Ln\sbull
Apt. 306\sbull
Foster City, CA

\spacedhrule{0.9em}{-0.4em}  % a horizontal line with some vertical spacing before and after

\roottitle{Summary}  % a root section title

\vspace{-1.3em}  % some vertical spacing
\begin{multicols}{2}  % open a multicolumn environment
\noindent \emph{Linux geek with a passion for developing maintainable software.}
\\
\\
My interest in Computer Science began in my first semester of college, when I took an introductory course on programming in \CPP. Since that course, every semester thereafter left me more and more captivated by the beauty of computing. Computer algorithms are truly captivating. Every problem breaks down into pieces that seem trivial on their own, but come together to form an awe-inspiring, cohesive whole which manages to keep itself stable. Simple pieces come together to compose elaborate, complex, mechanical structures which form the foundation of our modern society.

Since my introductory course, my interests have branched into many other specialized topics. Particular topics which interest me (but which I do not claim to be an expert in) are: graph theory, cryptography, discrete optimization, parallelization, and systems programming.

Thus far, I have completed one six-month internship at a San Francisco-based startup called CloudPassage, and am currently enjoying my second internship at OpenDNS.
\end{multicols}


\spacedhrule{0em}{-0.4em}

\roottitle{Experience}

\headedsection  % sets the header for a subsection and contains usually body text
  {\href{http://www.opendns.com}{OpenDNS}}
  {\textsc{San Francisco, CA}} {%
  \headedsubsection
    {Software Engineering Intern}
    {Sept 2014 -- present}
    {\bodytext{Assisted building a new backend framework that manages a graph database of malicious URLs/domains, primarily in Java. Also assisted in building front-end tooling with Python, which included a framework for managing various internal and external feeds which fed the backend databases. Also included writing integration testing frameworks and creating test environments with Vagrant and Puppet.}}
}

\headedsection  % sets the header for a subsection and contains usually body text
  {\href{http://www.opendns.com}{OpenDNS}}
  {\textsc{San Francisco, CA}} {%
  \headedsubsection
    {Security Research Intern}
    {Apr 2014 -- Sept 2014}
    {\bodytext{Assisted in improving security analysis tools which reported various information about OpenDNS's security graph. Some examples include a Python library, \href{https://github.com/dead10ck/semanticnet}{semanticnet}, which wraps and extends a graph library for use in our 3D graph visualization tool, \href{https://github.com/ThibaultReuille/graphiti}{OpenGraphiti}; additions to our internal web apps which query our malware database; a command line tool for querying our internal APIs; and various other tools.}}
}

\headedsection  % sets the header for the section and includes any subsections
  {\href{http://www.cloudpassage.com}{CloudPassage}}
  {\textsc{San Francisco, CA}} {%
  \headedsubsection
    {Security Research Intern}
    {Oct 2013 -- Apr 2014}
    {\bodytext{Assisted the security researchers in automating common manual work, such as detecting false positives in CVE vulnerability detection. Created tools to help security analysts detect system changes upon software installation by footprinting key resources before and after installation---used by analysts to create new security policies. Created a web app used by analysts to help ensure system configuration policies adhere to national security mandates (from NIST, HIPAA, etc.).}}
}

\vspace{-0.2em}
\begin{center}
  \emph{\small Please refer to my \href{http://www.linkedin.com/in/skylerhawthorne}{Linked-in profile} for a more complete list of work experiences along with recommendations.}
\end{center}


\spacedhrule{-0.2em}{-0.4em}

\roottitle{Education}

\headedsection
  {\href{http://www.sjsu.edu/}{San Jos\'{e} State University}}
  {\textsc{San Jos\'{e}, CA}} {%
  \headedsubsection
    {Bachelor degree in Computer Science}
    {2013 -- 2015 (expected)}
    {}
}

\headedsection
  {\href{http://www.canyons.edu/}{College of the Canyons}}
  {\textsc{Santa Clarita, CA}} {%
  \headedsubsection
    {Associate degree in Computer Science}
    {2009 -- 2013} {}
}

\spacedhrule{0.5em}{-0.4em}

\roottitle{Skills}

\inlineheadsection  % special section that has an inline header with a 'hanging' paragraph
  {Technical expertise:}
  {Software design and implementation. Big fan of open-source software, and prefers Linux in all environments, both for development and personal use. Very comfortable in the terminal, and proficient with git. Proficient with Python, Java, \CPP, Go, and Rust.}

\pagebreak

\roottitle{Related coursework}

\begin{multicols}{2}
	\begin{description}
		\item[\hspace{1em} San Jos\'{e} State University / College of the Canyons]{ \hfill \\[-1.5em]
		\begin{itemize}
			\item{Data Structures and Algorithms}
			\item{Parallel Processing}
			\item{Quantum Computing}
			\item{Microcontrollers and Assembly Language Programming}
			\item{Object-Oriented Programming in \CPP}
		\end{itemize}
		}
	\end{description}
	\begin{description}
		\item[\hspace{1em} Coursera]{ \hfill \\[-1.5em]
		\begin{itemize}
			\item{The Hardware/Software Interface (i.e. computer architecture, x86 assembly)}
			\item{Cryptography I}
			\item{Discrete Optimization}
		\end{itemize}
		}
	\end{description}
\end{multicols}

\end{document}
