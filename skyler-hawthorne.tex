%
% LaTeX source of my resume
% =========================
%
% Heavily commented to to fit even LaTeX beginners (hopefully).
%
% See the `README.md` file for more info.
%
% This file is licensed under the CC-NC-ND Creative Commons license.
%


% Start a document with the here given default font size and paper size.
\documentclass[10pt]{article}

% Set the page margins.
\usepackage[margin=0.75in]{geometry}

% Setup the language.
\usepackage[english]{babel}
\hyphenation{Some-long-word}

% Makes resume-specific commands available.
\usepackage{resume}

\usepackage[utf8]{inputenc}

\begin{document}  % begin the content of the document
\sloppy  % this to relax whitespacing in favour of straight margins


% title on top of the document
\maintitle{Skyler Hawthorne}{}{Last update on \today}

\nobreakvspace{0.3em}  % add some page break averse vertical spacing

% \noindent prevents paragraph's first lines from indenting
% \mbox is used to obfuscate the email address
% \sbull is a spaced bullet
% \href well..
% \\ breaks the line into a new paragraph
\noindent\href{mailto:skyler.at.dead10ck.dot.com}{skyler\mbox{}@\mbox{}dead10ck.com}\sbull
(408) 442-1637\sbull
\href{http://www.linkedin.com/in/skylerhawthorne}{www.linkedin.com/in/skylerhawthorne}
\\
88 North Jackson Ave\sbull
Unit 124\sbull
San José, CA\sbull
95116

\spacedhrule{0.9em}{-0.4em}  % a horizontal line with some vertical spacing before and after

\roottitle{Experience}

\headedsection  % sets the header for a subsection and contains usually body text
  {\href{http://www.opendns.com}{OpenDNS}}
  {\textsc{San Francisco, CA}} {%
  \headedsubsection
    {Software Engineer}
    {Sept 2014 -- present}
    {\bodytext{Worked on a team that managed a distributed graph database that represented the state of the internet as we knew it, including security-related information, like which URLs/domains were malicious. The work was primarily in Java, with my individual focus being on processing streams of updates to the database, in a highly distributed and parallel manner. \\

Additional responsibilities included: building middle-layer REST APIs that query the backend for user-facing data; building surrounding tooling with Python, which included a framework for managing various internal and external feeds which fed the update streams; writing integration tests and creating test environments with Docker; managing all our infrastructure in AWS to support all of the above work—which included high availability Hadoop/HBase, Kafka, Zookeeper, and ElasticSearch clusters, individual EC2 instances, and many other AWS resources—with Ansible, Terraform, and various other plumbing.\\

Started as a part-time intern, and was hired full-time in May 2015. \\

Technologies used: Java \sbull Python \sbull HBase \sbull Kafka \sbull ElasticSearch \sbull ZooKeeper \sbull Docker \sbull Puppet \sbull Ansible \sbull Terraform}}
}

\headedsection  % sets the header for a subsection and contains usually body text
  {\href{http://www.opendns.com}{OpenDNS}}
  {\textsc{San Francisco, CA}} {%
  \headedsubsection
    {Security Research Intern}
    {Apr 2014 -- Sept 2014}
    {\bodytext{Assisted in improving security analysis tools which reported various information about OpenDNS's security graph. Some examples include a Python library, \href{https://github.com/dead10ck/semanticnet}{semanticnet}, which wraps and extends a graph library for use in our 3D graph visualization tool, \href{https://github.com/ThibaultReuille/graphiti}{OpenGraphiti}; additions to our internal web apps which query our malware database; a command line tool for querying our internal APIs; and various other tools.}}
}

\headedsection  % sets the header for the section and includes any subsections
  {\href{http://www.cloudpassage.com}{CloudPassage}}
  {\textsc{San Francisco, CA}} {%
  \headedsubsection
    {Security Research Intern}
    {Oct 2013 -- Apr 2014}
    {\bodytext{Assisted the security researchers in automating common manual work, such as detecting false positives in CVE vulnerability detection. Created tools to help security analysts detect system changes upon software installation by footprinting key resources before and after installation---used by analysts to create new security policies. Created a web app used by analysts to help ensure system configuration policies adhere to national security mandates (from NIST, HIPAA, etc.).}}
}

\vspace{-0.2em}
\begin{center}
  \emph{\small Please refer to my \href{http://www.linkedin.com/in/skylerhawthorne}{Linked-in profile} for a more complete list of work experiences along with recommendations.}
\end{center}


\spacedhrule{-0.2em}{-0.4em}

\roottitle{Education}

\headedsection
  {\href{http://www.sjsu.edu/}{San Jos\'{e} State University}}
  {\textsc{San Jos\'{e}, CA}} {%
  \headedsubsection
    {Bachelor degree in Computer Science}
    {2013 -- 2015}
    {}
}

\headedsection
  {\href{http://www.canyons.edu/}{College of the Canyons}}
  {\textsc{Santa Clarita, CA}} {%
  \headedsubsection
    {Associate degree in Computer Science}
    {2009 -- 2013}
    {}
}

\spacedhrule{0.5em}{-0.4em}

\roottitle{Skills}

\inlineheadsection  % special section that has an inline header with a 'hanging' paragraph
  {Technical expertise:}
  {Software design and implementation. Big fan of open-source software, and prefers Linux in all environments, both for development and personal use. Very comfortable in the terminal, and proficient with git. Proficient with Java, Python, and Rust. Was proficient with Go in the past. Basic skills with C and \CPP.}

\roottitle {Web pages:}

\href{https://github.com/dead10ck}{GitHub}

\end{document}
